% \documentclass{beamer}
\documentclass[handout]{beamer}

\usepackage{beamerthemesplit}
\usepackage[latin1]{inputenc}
\usepackage[brazilian]{babel}
\PassOptionsToPackage{pdftex}{graphicx}
\usepackage{epstopdf}

%cores
%http://deic.uab.es/~iblanes/beamer_gallery/index_by_theme_and_color.html


\usepackage{beamerthemeshadow}
%\usepackage{graphicx} % Enable the use of graphs
\usepackage{amsfonts} % Enable special mathemtical characters.
\usepackage{amsmath}  % Enable the use  Multi-line equations, compound symbols, etc
%\usepackage{extsizes} % Enable the use of different font sizes. DANDO PAU
\usepackage{booktabs} % More complex tables
\usepackage{hyperref} % Links
\usepackage{subfig}   % Subfigures
\usepackage{helvet}   % Different fonts
% Need also to download pgf, ms and xcolor packages.
%\usepackage[ansinew]{inputenc} % Permite digitar os acentos diretamente.
%\usepackage[portuguese]{babel} % Habilita a l�ngua portuguesa

\usetheme{Berlin} % Options: default, Antibes, Berkeley, Berlin, Copenhagen, Madrid, Pittsburgh, Singapore, etc.
%\useinnertheme{default} % Options: default, circles, rectangles, rounded, inmargin,
%\useoutertheme{default} % Options: default, infolines, miniframes, etc. Options []: footline=empty, footline=authorinstitute, footline=authortitle, footline=institutetitle, footline=authorinstitutetitle, subsection=.true or false.

%\usecolortheme{default} % Options: default, beetle, albatross, crane, fly, wolverine, beaver, whale, etc.
%\useinnertheme{orchid} % Inner Color Themes: lily, orchid, rose
%\useoutertheme{dolphin} % Outer Color Themes: whale, seahorse, dolphin

\setbeamersize{text margin left=0.6cm, text margin right=0.6cm}

\setbeameroption{hide notes} % Use this option instead in presentations with projectors: show notes on second screen
% to add notes, type: \note[item]<1>{Text}









%%%%%%%%%%%%%%
\usepackage{listings}
\usepackage{color}
\definecolor{dkgreen}{rgb}{0,0.6,0}
\definecolor{gray}{rgb}{0.5,0.5,0.5}
\definecolor{mauve}{rgb}{0.58,0,0.82}
\definecolor{bg}{rgb}{0.95,0.95,0.95}
\lstset{ 
  language=Octave,                % the language of the code
  basicstyle=\footnotesize,       % the size of the fonts that are used for the code
  numbers=left,                   % where to put the line-numbers
  numberstyle=\tiny\color{gray},  % the style that is used for the line-numbers
  stepnumber=1,                   % the step between two line-numbers. If it's 1, each line 
                                  % will be numbered
  numbersep=5pt,                  % how far the line-numbers are from the code
  %backgroundcolor=\color{bg},     % choose the background color. You must add \usepackage{color}
  showspaces=false,               % show spaces adding particular underscores
  showstringspaces=false,         % underline spaces within strings
  showtabs=false,                 % show tabs within strings adding particular underscores
  frame=single,                   % adds a frame around the code
  rulecolor=\color{black},        % if not set, the frame-color may be changed on line-breaks within not-black text (e.g. commens (green here))
  tabsize=4,                      % sets default tabsize to 2 spaces
  captionpos=b,                   % sets the caption-position to bottom
  breaklines=true,                % sets automatic line breaking
  breakatwhitespace=false,        % sets if automatic breaks should only happen at whitespace
  title=\lstname,                 % show the filename of files included with \lstinputlisting;
                                  % also try caption instead of title
  keywordstyle=\color{blue}, %blue     % keyword style
  commentstyle=\color{dkgreen}, %dkgreen   % comment style
  stringstyle=\color{mauve}, %mauve      % string literal style
  escapeinside={\%*}{*)},         % if you want to add a comment within your code
  morekeywords={*,...}            % if you want to add more keywords to the set
}

\usepackage{hyperref}
\usepackage{ulem}












\setbeamertemplate{headline}
{%
	\begin{beamercolorbox}[colsep=1.5pt]{upper separation line head}
	\end{beamercolorbox}
%	\begin{beamercolorbox}{section in head/foot}
%		\vskip2pt\insertnavigation{\paperwidth}\vskip2pt
%	\end{beamercolorbox}%
	\begin{beamercolorbox}[colsep=1.5pt]{lower separation line head}
	\end{beamercolorbox}
}







\begin{document}

\title{SIMO: Uma Ferramenta Online para Auxiliar o Ensino de Otimiza��o}
\author{Elias Luiz da Silva J�nior}
% \date{Novembro de 2011}
\date{}
%\date{\today} 
\institute{SBPO 2016}
%\logo{\includegraphics[scale=0.12]{images/_logoBRL}}

\begin{frame}
\titlepage
\end{frame}

\begin{frame}\frametitle{Autores}
	\begin{tabular}{ll}
		Andr� R. da Cruz (UFMG) & \url{andrecruz@timoteo.cefetmg.br}\\ \\
		Andr� F. R. Malta (CEFET-MG) & \url{andrmalta@gmail.com}\\ \\
		Daniel G. Oliveira (CEFET-MG) & \url{daniel.gdo@hotmail.com}\\ \\
		Elias L. S. J�nior (CEFET-MG) & \url{eliasluizjr@gmail.com}\\ \\
		Jo�o M. M. C. Cota (CEFET-MG) & \url{joao\_marcos\_cota@hotmail.com}\\ \\
	\end{tabular}
\end{frame}

\begin{section}{Introdu��o}

\begin{frame}\frametitle{O que � o SIMO}
	\begin{itemize}
		\item SIMO: Sistema Interativo para M�todos de Otimiza��o
		\bigskip
		\item Plataforma online para a execu��o interativa de algoritmos de otimiza��o
		\bigskip
		\item Explica��es sobre os algoritmos dispon�veis
	\end{itemize}
\end{frame}

\begin{frame}\frametitle{Motiva��o}
	\begin{itemize}
		\item Algoritmos complexos e com casos especiais
		\bigskip
		\item \textit{Solvers} apresentam apenas o resultado final
	\end{itemize}
\end{frame}

\begin{frame}\frametitle{Objetivos}
	\begin{itemize}
		\item Interface simples e intuitiva
		\bigskip
		\item Visualizar a execu��o dos algoritmos
		\bigskip
		\item Interagir com os mesmos
	\end{itemize}
\end{frame}

\end{section}
\begin{section}{Conte�do}

\begin{frame}\frametitle{Conte�do abordado}
	\begin{itemize}
		\item Simplex
		\bigskip
		\item \textit{Branch and Bound}
		\bigskip
		\item Algoritmos de transporte
	\end{itemize}
\end{frame}

\begin{frame}\frametitle{Simplex}
	\begin{itemize}
		\item Algoritmo para programa��o linear
		\bigskip
		\item Variantes abordadas:
		\begin{itemize}
			\medskip
			\item Grande M
			\medskip
			\item Duas Fases
			\medskip
			\item Generalizado
		\end{itemize}
	\end{itemize}
\end{frame}

\begin{frame}\frametitle{\textit{Branch-and-Bound}}
	\begin{itemize}
		\item Algoritmo para programa��o linear inteira
		\bigskip
		\item Gera uma �rvore de poss�veis modelos, resolvidos pelo simplex
	\end{itemize}
\end{frame}

\begin{frame}\frametitle{Algoritmos de Transporte}
	\begin{itemize}
		\item Aloca��o de recursos
		\bigskip
		\item Mais simples que a resolu��o do modelo via simplex
		\bigskip
		\item Algoritmos abordados:
		\begin{itemize}
			\medskip
			\item Canto noroeste
			\medskip
			\item Menor custo
			\medskip
			\item M�todo da aproxima��o de Vogel
		\end{itemize}
	\end{itemize}
\end{frame}

\end{section}
\begin{section}{Tecnologias}

\begin{frame}\frametitle{Web}
	\begin{itemize}
		\item Desenvolvido para web
		\bigskip
		\item Dispon�vel para qualquer plataforma com acesso � internet
	\end{itemize}
\end{frame}

\begin{frame}\frametitle{Tecnologias}
	\begin{itemize}
		\item Desenvolvido completamente em Javascript
		\bigskip
		\item Executando na m�quina do usu�rio
	\end{itemize}
\end{frame}

\begin{frame}\frametitle{Tecnologias}
	Foram utilizadas algumas bibliotecas:
	\bigskip
	\begin{itemize}
		\medskip
		\item Bootstrap
		\medskip
		\item Vis
		\medskip
		\item Mathjax
		\medskip
		\item GLPK
	\end{itemize}
\end{frame}

\begin{frame}\frametitle{Github}
  \begin{itemize}
  	\item Pode ser utilizado no site http://jmmccota.github.io/Otimizacao
  	\bigskip
  	\item C�digo aberto https://github.com/jmmccota/Otimizacao
  \end{itemize}
\end{frame}

\end{section}

\end{document}
